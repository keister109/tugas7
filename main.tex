\documentclass[conference]{IEEEtran}
\usepackage{graphicx}
\usepackage{amsmath}

\graphicspath{{./gambar/}}

\title{Analisis kekuatan}

\author{\IEEEauthorblockN{andria\IEEEauthorrefmark{1}, hani\IEEEauthorrefmark{2}}
\IEEEauthorblockA{\textit{Faculty of Information Technology}\\
\textit{Institut Teklnologi Batam}\\
Batam, Indonesia\\
Email: }}

\begin{document}

\maketitle

\begin{abstract}
    abstrak
\end{abstract}

\begin{IEEEkeywords}
    keywords
\end{IEEEkeywords}

\section{Introduction}

\section{Related Work}

\section{Scenario}

\begin{figure}[htbp]
    \input{gambar/topologi.pdf_tex}
\end{figure}

\section{Hasil dan Pembahasan}

\begin{equation}
    Rerata RSSI = \frac{Total Jumlah Nilai RSSI}{Jumlah Koordinat receiver}
    \label{rerata_rssi}
\end{equation}

berdasarkan persamaan~\ref{rerata_rssi}

\section{Kesimpulan}

\bibliographystyle{IEEtran}
\bibliography{referensi.bib}

\end{document}