\documentclass[conference]{IEEEtran}
\usepackage{graphicx}
\usepackage{amsmath}

\graphicspath{{./gambar/}}

\title{Analisis kekuatan Sinyal Menggunakan inSSIDer}

\author{Kevin Antony K\IEEEauthorrefmark{1}, Maranti Nainggolan\IEEEauthorrefmark{2}\\
\textit{Fakultas Teknologi Informasi}\\
\textit{Teknik Komputer}\\
\textit{Institut Teknologi Batam}\\
Batam, Indonesia\\
Email: \{\IEEEauthorrefmark{1}1922003, \IEEEauthorrefmark{2}1922023\}@student.iteba.ac.id}

\begin{document}

\maketitle

\begin{abstract}
    
Kemajuan teknologi informasi pada saat ini
terus berkembang seiring dengan kebutuhan manusia yang
menginginkan kemudahan, kecepatan, dan keakuratan dalam
memperoleh informasi. Oleh karena itu kemajuan teknologi
informasi di bidang transmisi pada saat ini yang berkembang
selain fiber optic ialah penggunaan perangkat wireless. Perangkat
wireless ini memungkinkan adanya hubungan para pengguna
informasi dalam melakukan aktivitasnya
\end{abstract}

\begin{IEEEkeywords}
Access point, InSSIDer, SSID, Wi-Fi.
\end{IEEEkeywords}

\section{Introduction}
Teknologi Wifi atau yang dikenal dengan wireless LAN(WLAN) telah banyak diimplemantasikan 
oleh masyarakat baik didalam maupun diluar negeri.Selain untuk aplikasi privat,WLAN 
juga banyak digunakan untuk aplikasipublic(hospot).\

\vspace{0.2cm}

WLAN merupakan jaringan yang tidak tampak karena
merupakan gelombang radio. Terutama bila frekuensinya terlalu
berdekatan, atau hilang oleh daya gelombang radio yang
lebih besar sehingga jaringan yang kita buat menjadi tidak
efisien. Untuk itu diperlukan suatu software yang dapat digunakan
untuk mencari informasi jaringan WLAN pada suatu
area lebih spesifik dari scan biasa. Salah satu software yang
dapat digunakan adalah inSSIDer.

Kemajuan teknologi informasi pada saat ini terus berkembang seiring dengan kebutuhan manusia yang menginginkan kemudahan, kecepatan, dan keakuratan dalam memperoleh informasi. 
Oleh karena itu kemajuan teknologi informasi khususnya dibidang teknologi jaringan nirkabel atau sering kita kenal dengan Wireless.

\section{Related Work}
Pada Pembahasan ini, akan ditampilkan hasil analisis kekuatan sinyal Wi-Fi pada suatu perubhan dengan Menggunakan
software inSSIDer. kita sebagai administrator jaringan yang akan melakukan instalasi jaringan WLAN(hospot atau Wifi)
baru pada suatu wilayah , tentunya sebelum melakukan instalasi dan perancangan jaringan maka perlu
dilakukan survey terhadap wilayah yang akan menjadi sasaran pemasangan jaringan tersebut
\section{Scenario}
1.Wireless Personal Area Network (WPAN)
WPAN (Wireless Personal Area Network) adalah sebuah bentuk komunikasi wireless yang terbatas hanya pada jarak pendek dan umumnya hanya terbatas untuk dua buah perangkat elektronik.
\vspace{5pt}

2.Wireless Wide Area Network (WWAN)
WWAN adalah sebuah bentuk komunikasi nirkabel yang memiliki area sangat luas, antara lain untuk penggunaan selular seperti 2G, 3G, 4G, dan lain sebagainya
\vspace{5pt}

3.Wireless Local Area Network (WLAN)
WLAN (Wireless Local Area Network) adalah sebuah bentuk komunikasi nirkabel yang memiliki area terbatas seperti dalam suatu ruangan ataupun sebuah gedung. WLAN memiliki standar komunikasi yang diatur oleh sebuah lembaga. Standar komunikasi data yang digunakan dalam WLAN umumnya adalah keluarga Institute of Electrical and Electronics Engineers (IEEE) 802.11.
\begin{itemize}
    \item  IEEE 802.11a bekerja pada frekuensi 5GHz dan mempunyai kecepatan maksimum 54 Mbps.
    \item IEEE 802.11b bekerja pada frekuensi 2,4GHz dan mempunyai kecepatan sampai dengan 11Mbps.
    \item IEEE 802.11g bekerja pada frekuensi yang sama dengan IEEE 802.11b yaitu 2,4GHz, namun memiliki kecepatan maksimal yang lebih besar, yaitu 54Mbps.
    \item   IEEE 802.11n yang bekerja pada dua frekuensi yaitu 2,4 dan 5GHz dengan kecepatan maksimum adalah 100 sampai dengan 210 Mbps
\end{itemize}
\vspace{5pt}

4.Wireless MAN (WMAN)
\vspace{1pt}

MAN atau Metropolitan Area Network mencakup area yang lebih besar daripada LAN dan area yang lebih kecil dibandingkan dengan WAN.
\subsection{Perhitungan Kecepatan Internet}
kecepatan akses internet adalah kecepatan transfer data pada saat melakukan akses melalui jalur internet. Kecepatan transfer data adalah jumlah data dalam bit yang melewati suatu media tertentu dalam satu detik.
Jadi kalau ditulis dengan rumus, Kecepatan Internet (transfer data) dapat dihitung dengan rumus:
\begin{equation}
    Kecepatan = \frac{Jumlah File Data}{Waktu}
    \label{rerata_rssi}
\end{equation}

berdasarkan persamaan~\ref{rerata_rssi} kita dapat mengukur dan menghitung 
kecepatan internet secara manual atau Menggunakan rumus 

\begin{figure}[htbp]
    \input{gambar/topologi.pdf_tex}
\end{figure}

\section{Hasil dan Pembahasan}



\section{Kesimpulan}

\bibliographystyle{IEEtran}
\bibliography{referensi.bib}

\end{document}