\documentclass[conference]{IEEEtran}
\usepackage{graphicx}
\usepackage{amsmath}

\graphicspath{{./gambar/}}

\title{Analisis kekuatan Sinyal Menggunakan inSSIDer}

\author{Kevin Antony K\IEEEauthorrefmark{1}, Maranti Nainggolan\IEEEauthorrefmark{2}\\
\textit{Fakultas Teknologi Informasi}\\
\textit{Teknik Komputer}\\
\textit{Institut Teknologi Batam}\\
Batam, Indonesia\\
Email: \{\IEEEauthorrefmark{1}1922003, \IEEEauthorrefmark{2}1922023\}@student.iteba.ac.id}

\begin{document}

\maketitle

\begin{abstract}
    
    Kemajuan teknologi informasi pada saat ini
terus berkembang seiring dengan kebutuhan manusia yang
menginginkan kemudahan, kecepatan, dan keakuratan dalam
memperoleh informasi. Oleh karena itu kemajuan teknologi
informasi di bidang transmisi pada saat ini yang berkembang
selain fiber optic ialah penggunaan perangkat wireless. Perangkat
wireless ini memungkinkan adanya hubungan para pengguna
informasi dalam melakukan aktivitasnya
\end{abstract}

\begin{IEEEkeywords}
    keywords
\end{IEEEkeywords}

\section{Introduction}
Teknologi Wifi atau yang dikenal dengan wireless LAN(WLAN) telah banyak diimplemantasikan 
oleh masyarakat baik didalam maupun diluar negeri.Selain untuk aplikasi privat,WLAN 
juga banyak digunakan untuk aplikasipublic(hospot).\

Kemajuan Teknologi Informasi pada saat ini juga 


\section{Related Work}

\section{Scenario}

\begin{figure}[htbp]
    \input{gambar/topologi.pdf_tex}
\end{figure}

\section{Hasil dan Pembahasan}

\begin{equation}
    Rerata RSSI = \frac{Total Jumlah Nilai RSSI}{Jumlah Koordinat receiver}
    \label{rerata_rssi}
\end{equation}

berdasarkan persamaan~\ref{rerata_rssi}

\section{Kesimpulan}

\bibliographystyle{IEEtran}
\bibliography{referensi.bib}

\end{document}